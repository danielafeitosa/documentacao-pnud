\documentclass[11pt]{article}
\renewcommand{\rmdefault}{ptm}
\usepackage[scaled=0.92]{helvet}
\usepackage{courier,xcolor,colortbl,listings,parskip,graphicx,fancyvrb,fancyhdr,lastpage}
\usepackage{float,framed}
\normalfont
\usepackage[T1]{fontenc}
\setlength{\parskip}{7pt}
\usepackage[toc,page]{appendix}
\usepackage[hmargin=2.5cm,vmargin=2cm]{geometry}
\usepackage[utf8]{inputenc}
\usepackage[brazil]{babel}
\pagestyle{fancy}
\setlength{\headheight}{120pt}
\setlength{\headsep}{30pt}
\setlength{\textheight}{550pt}
\renewcommand{\headrulewidth}{0pt}
\lhead{}
\rhead{}
\chead{\includegraphics{brasao.jpg}\\
        \large \textbf{PRESIDÊNCIA DA REPÚBLICA}\\
        \large SECRETARIA-GERAL\\
        \large Secretaria-Executiva}
\cfoot{}
\rfoot{\thepage /\pageref{LastPage}}
\hyphenation{par-ti-ci-pa-ção}
\bibliographystyle{ieeetr}

\newcommand{\MyName}{Daniela Soares Feitosa}
\newcommand{\MyEmail}{daniela@colivre.coop.br}
\newcommand{\ContractNumber}{2013/000292}
\newcommand{\ProjectCode}{Projeto PNUD BRA/12/018}
\newcommand{\NomeSecretaria}{Secretaria Geral da Presidência da República}
\newcommand{\SiglaSecretaria}{SG/PR}
\newcommand{\ProductNumber}{04}
\newcommand{\ProductDescription}{Documento com proposta para
desenvolvimento do código do aplicativo de consulta pública, do código
de integração dele com o portal e do painel de controle e administração,
contendo exemplos e códigos.}
\newcommand{\MesEntrega}{Setembro de 2013}
\newcommand{\DiaEntrega}{05}

\begin{document}
\lstset{language=Ruby}
\definecolor{light-gray}{gray}{0.95}
\lstdefinestyle{codeFrame}{backgroundcolor=\color{light-gray},frame=lines}

\textbf{\ProjectCode \ -} \ProductDescription

\vspace{3cm}

\begin{minipage}{0.5\textwidth}
  \textbf{Consultora: \MyName}
  \newline
  \textbf{Contrato nº: \ContractNumber}
  \newline
  \textbf{Produto / nº: \ProductNumber}
\end{minipage}

\vspace{2cm}

\textbf{Assinatura do Consultor}

\begin{framed}
Local e data: Brasília/DF, \DiaEntrega \ de \MesEntrega 
\newline
\newline
Assinatura do Consultor: \line(1,0){300}
\end{framed}

\vspace{1cm}

\textbf{Assinatura do Supervisor}

\begin{framed}
Atesto que os serviços foram prestados conforme estabelecido no Contrato
de Consultoria.
\newline
\newline
Local e data: Brasília/DF, \DiaEntrega \ de \MesEntrega 
\newline
\newline
Assinatura e Carimbo: \line(1,0){300}
\end{framed}

\clearpage
\newcolumntype{g}{>{\columncolor{light-gray}}l}

\begin{center}
  \begin{tabular}{| g | p{10cm} |}
    \hline
    \textbf{Título} & \ProductDescription \\ \hline
    \textbf{Língua do documento} & Português - Brasil \\ \hline
    \textbf{Documentação de referência} & Português \\ \hline
    \textbf{Unidade responsável} & \NomeSecretaria \
(\SiglaSecretaria) \\ \hline
    \textbf{Criador} & \MyName - \MyEmail \\ \hline
    \textbf{Taxonomias} & Desenvolvimento \\ \hline
    \textbf{Data de aprovação} &  \\ \hline
    \textbf{Público} & \SiglaSecretaria, Parceiros e Sociedade
Civil \\ \hline
    \textbf{Faz parte do} & \ProjectCode \\ \hline
    \textbf{Em conformidade com a} & \NomeSecretaria \\ \hline
    \textbf{Documentos anexos} & Página da Comunidade OSC - Organizações da
Sociedade Civil; CSS do cabeçalho do tema de comunidade; Código javascript para
incluir uma classe identificadora; Código CSS para estilizar os blocos com o padrão
colorido \\ \hline
    \textbf{Revisado em} &  \\ \hline
  \end{tabular}
\end{center}

\clearpage

\tableofcontents
\clearpage
\listoffigures

\clearpage

\section{Apresentação}

Em consonância com os objetivos e cronograma previsto no âmbito do
projeto BRA/12/018:
\textbf{Desenvolvimento de Metodologias de Articulação e Gestão de
Políticas Públicas para Promoção da Democracia Participativa},
firmado entre a Secretaria-Geral da Presidência da República
(SG/PR) e o Programa das Nações Unidas para o Desenvolvimento (PNUD),
o presente documento apresenta a proposta para
desenvonlvimento do código do aplicativo de consulta pública, do código
de integração dele com o portal e do painel de controle e
administração, contendo exemplos e códigos.

Essa proposta está configurada como produto 4 da consultoria técnica
para especificação da construção dos códigos das metodologias de
organização da informação e interação participativa do portal de
participação social.

Neste documento será apresentada a especificação e
modelagem do código para auxiliar os órgãos e entidades do
Governo Federal no relacionamento e articulação com os movimentos
sociais através do Portal de Consulta Pública, um dos canais que
possibilita a consulta e participação popular na discussão e na definição
da agenda prioritária do país.

Como o Portal de Consulta Pública utiliza o software livre Noosfero,
plataforma web para redes sociais, o código produzido deverá ser público
e divulgado para a comunidade e para os que desenvolvem e se utilizam do
software.

O código foi desenvolvido para ser utilizado como um plugin do Noosfero.
O plugin de Classificação de Comentários ( {\it CommentClassification} )
permite:

\begin{itemize}
  \item Gestão de {\it Etiquetas} e {\it Status} dos comentários;
  \item Habilitação da funcionalidade de Apoio ({\it Support}) à
comentários;
    \subitem Visualização da listagem de usuários que apoiam um comentário;
  \item Possibilidade de definir uma Etiqueta a um comentário;
  \item Possibilidade de adicionar {\it Status} e justificativas a um
comentário;
    \subitem Possibilidade de concordar com uma justificativa;
  \item Possibilidade de sugerir novas redações aos trechos disponíveis
na consulta pública;
  \item Gestão das possibilidades de classificação dos comentários por
perfil;
  \item Visualização das estatísticas das classificações;
  \item Exportação em CSV dos trechos, comentários, etiquetas, status,
justificativas e novas redações;
\end{itemize}

Como conteúdo deste documento também serão apresentados as telas
iniciais para interação com o plugin, incluindo instruções passo a
passo para a utilização do Portal de Consulta Pública tanto para usuário
visitante como administrador do ambiente.

\section{Plugin de Classificação de Comentários}

O Plugin de Classificação de Comentários permite algumas formas de
classificar um comentário: etiquetas, status e suporte.

O plugin foi desenvolvido para ser utilizado no Noosfero e pode ser
habilitado e desabilitado em qualquer instalação do Noosfero.

O código do plugin pode ser visto no Apêndice~\ref{App:PluginCode}

Para disponibilizar um plugin num ambiente do Noosfero, é necessário
primeiro habilitá-lo no servidor:

\begin{verbatim}

$ ./script/noosfero-plugins enable comment_classification

\end{verbatim}

E depois habilitar no ambiente. A tela de administração de plugins está
sendo mostrada na Figura~\ref{fig:environment-admin-page}.

\begin{figure}[h]
\center
\includegraphics[scale=0.6]{environment-admin-page.png}
\caption{Tela de administração dos plugins num ambiente Noosfero}
\label{fig:environment-admin-page}
\end{figure}

Abaixo da descrição do plugin o administrador do ambiente poderá
gerenciar as classificações, clicando no link Configuração que pode ser
visto na Figura~\ref{fig:environment-admin-page}. A tela representada na
Figura~\ref{fig:plugin-admin-page} mostra os links de administração das
Etiquetas e Status.

\begin{figure}[h]
\center
\includegraphics[scale=0.6]{plugin-admin-page.png}
\caption{Tela de administração do plugin}
\label{fig:plugin-admin-page}
\end{figure}

\section{Etiquetas}

As etiquetas permitem a classificação única dos comentários.

\subsection{Definição}

Foi definida uma nova classe {\it Label}
(Apêndice~\ref{App:PluginLabel}) no Plugin de classificação de
comentários para a inclusão da funcionalidade de Classificação por
Etiquetas. Cada comentário pode ser classificado por uma Etiqueta, que possui os
seguintes atributos:
\begin{itemize}
  \item Nome ({\it name}): cada etiqueta terá um nome único, que a representará no
ambiente;
  \item Cor ({\it color}): será utilizada ao mostrar a etiqueta de um comentário. As
cores permitem que uma informação seja absorvida pelo leitor de forma
mais rápida, então a escolha das cores deve ter alguma relação com a
ideia que a Etiqueta deve passar;
  \item Habilitado ({\it enabled}): As etiquetas podem ser habilitadas e desabilitadas
pelos administradores. Apenas as etiquetas habilitadas estarão
disponíveis para classificação dos comentários;
  \item Dono ({\it owner}): Relaciona a etiqueta com o tipo que a criou.
As etiquetas podem ser definidas no ambiente e
nos perfis. As etiquetas definidas no contexto do ambiente ficarão
disponíveis para todos os perfis da rede e as definidas no contexto do
perfil ficarão disponíveis apenas para os perfis.
\end{itemize}

O sistema registra quando um usuário adiciona uma Etiqueta a um
comentário. A classe apresentada no
Apêndice~\ref{App:PluginCommentLabelUser} mostra as relações e
validações da relação. Quando um usuário classifica um comentário é
criada a relação no banco de dados com os seguintes atributos:
\begin{itemize}
  \item Profile ({\it profile}): Referencia o perfil que definiu a
etiqueta;
  \item Comment ({\it comment}): Referencia o comentário classificado;
  \item Label ({\it label}): Referencia a etiqueta definida;
\end{itemize}

As definições dessas classes e seus atributos podem ser vistas na
Figura~\ref{fig:labels-model}.

\begin{figure}[h]
\center
\includegraphics[scale=0.5]{labels-model.png}
\caption{Modelagem das classes para definir uma etiqueta a um comentário}
\label{fig:labels-model}
\end{figure}

O próprio proponente do comentário pode classificar inicialmente o
comentário e o avaliador pode alterar a classificação do proponente. Os
avaliadores são usuários do ambiente com perfil de moderação de
comentários no perfil.

\subsection{Administração}

Para a gestão de {\it Etiquetas} dos comentários foi criada uma tela com
a listagem das etiquetas disponíveis no ambiente
(Figura~\ref{fig:plugin-label-admin}). Nessa página o administrador
do ambiente poderá criar, editar e remover qualquer uma das etiquetas
listadas.

\begin{figure}[h]
\center
\includegraphics[scale=0.5]{plugin-label-admin.png}
\caption{Tela de gestão das Etiquetas}
\label{fig:plugin-label-admin}
\end{figure}

O formulário de criação de Etiquetas pode ser visto na
Figura~\ref{fig:new-label-page}. Nessa página o administrador
do ambiente poderá definir os atributos da etiqueta.

\begin{figure}[h]
\center
\includegraphics[scale=0.5]{new-label-page.png}
\caption{Criação de etiquetas}
\label{fig:new-label-page}
\end{figure}


\subsection{Utilização}

No formulário de criação de comentários do Noosfero foi adicionado um
campo extra que permite que um usuário escolha uma Etiqueta para
classificar seu comentário (Figura~\ref{fig:comment-add}).

Cada comentário poderá ter apenas uma etiqueta. As etiquetas serão
utilizadas para ajudar no processo de consulta e participação popular,
então é importante que a Etiqueta definida para um comentário realmente
descreva o tipo de comentário. Para garantir isso, o plugin permite que
pessoas com permissão de moderação de comentários, os {\it Avaliadores},
alterem a Etiqueta do comentário. A alteração é registrada no sistema e
é possível saber qual usuário realizou a mudança, permitindo uma
auditoria posterior.

\begin{figure}[h]
\center
\includegraphics[scale=0.5]{comment-add.png}
\caption{Escolha de etiquetas}
\label{fig:comment-add}
\end{figure}


Todas as pessoas com permissão de visualizar o comentário poderão
visualizar a Etiqueta definida para o comentário
(Figura~\ref{fig:comment-view-label}). A cor definida pela Etiqueta
ajudará aos visualizadores dos comentários a identificar rapidamente o
tipo de comentário.

\begin{figure}[h]
\center
\includegraphics[scale=0.6]{comment-view-label.png}
\caption{Visualização de um comentário com Etiqueta}
\label{fig:comment-view-label}
\end{figure}

\section{Status}

O Status permite a classificação dos comentários por pessoas com
permissões específicas.


\subsection{Definição}

Foi definida uma nova classe {\it Status}
(Apêndice~\ref{App:PluginStatus}) no Plugin de classificação de
comentários para a inclusão da funcionalidade de Classificação por
Status. Os comentários podem ser classificados por vários Status, que possui os
seguintes atributos:
\begin{itemize}
  \item Nome ({\it name}): cada status terá um nome único, que o representará no
ambiente;
  \item Habilitado ({\it enabled}): Os status podem ser habilitados e
desabilitados pelos administradores. Apenas os status habilitados estarão
disponíveis para classificação dos comentários;
  \item Habilita justificativa ({\it enable\_reason}): Os status podem
permitir a inclusão de uma justficativa;
 \item Dono ({\it owner}): Relaciona o status com o usuário que o
definiu. 
\end{itemize}

O sistema registra quando um usuário adiciona um Status a um
comentário. A classe apresentada no
Apêndice~\ref{App:PluginCommentStatusUser} mostra as relações e
validações da relação. Quando um usuário com permissão de moderação de
comentários classifica um comentário é criada a relação no banco de
dados com os seguintes atributos:
\begin{itemize}
  \item Profile ({\it profile}): Referencia o perfil que definiu o
status;
  \item Comment ({\it comment}): Referencia o comentário classificado;
  \item Status ({\it status}): Referencia o status definido;
  \item Reason ({\it reason}): Permite que seja inserida uma
justificativa para o Status escolhido;
\end{itemize}

A definição dessas classes e seus atributos podem ser vistas na
Figura~\ref{fig:status-model}.

\begin{figure}[h]
\center
\includegraphics[scale=0.5]{status-model.png}
\caption{Modelagem das classes para definir um status a um comentário}
\label{fig:status-model}
\end{figure}

Os status permitem que pessoas com permissão de moderação de
comentários, os {\it avaliadores}, definam o Status de um comentário e a
justificativa para tal escolha. Cada status definido é registrado no
sistema como um {\it log} de atividades e pode ser visualizado por
pessoas com permissões específicas.

\subsection{Administração}

Para a gestão de {\it Status} dos comentários foi criada uma tela com
a listagem dos status disponíveis no ambiente
(Figura~\ref{fig:plugin-status-admin}). Nessa página o administrador
do ambiente poderá criar, editar e remover qualquer um dos status
listados.

\begin{figure}[h]
\center
\includegraphics[scale=0.5]{plugin-status-admin.png}
\caption{Tela de gestão dos Status}
\label{fig:plugin-status-admin}
\end{figure}

\begin{figure}[h]
\center
\includegraphics[scale=0.5]{new-status-page.png}
\caption{Criação de status}
\label{fig:new-status-page}
\end{figure}

O formulário de criação de Status pode ser visto na
Figura~\ref{fig:new-status-page}. Nessa página o administrador
do ambiente poderá definir os atributos do status.

\subsection{Utilização}

Na visualização de um comentário do Noosfero foi adicionado um link para
a inclusão de um Status (Figura~\ref{fig:comment-view-status}). Esse
link só será visualizado pelos usuários com permissão de moderação de
comentários.

\begin{figure}[h]
\center
\includegraphics[scale=0.6]{comment-view-status.png}
\caption{Visualização de um comentário com link para adicionar Status}
\label{fig:comment-view-status}
\end{figure}

Cada comentário poderá ter várias justificativas. Cada usuário com
permissão de moderação de comentários poderá incluir várias
justificativas. A ideia é que cada Avaliador leia o histórico de
status e justificativas dos outros Avaliadores e que ocorra uma
discussão até a decisão final sobre o que será feito sobre o trecho
comentado (Figura~\ref{fig:new-status-page-and-history}).

Os status também serão utilizados para ajudar no processo de
consulta e participação popular, então é importante que o histórico e o
debate permitam uma classificação final do Status do comentário. Cada
inclusão de justificativa é registrada no sistema e não pode ser
alterada. Dessa forma é possível saber qual usuário definiu o status,
sua justificativa e a sugestão de novo texto, permitindo uma
auditoria posterior.

\begin{figure}[h]
\center
\includegraphics[scale=0.5]{new-status-page-and-history.png}
\caption{Escolha de status}
\label{fig:new-status-page-and-history}
\end{figure}

\section{Apoio}

A funcionalidade Apoio ({\it Support}) permite que usuários concordem com
um comentário.

\subsection{Definição}

Foi definida uma nova classe {\it Support}
(Apêndice~\ref{App:PluginSupport}) no Plugin de classificação de
comentários para a inclusão da funcionalidade de Classificação por
Apoio. O sistema registra quando um usuário apoia um comentário. Quando
um usuário demonstra sua concordância com outro comentário é criada a
relação no banco de dados com os seguintes atributos:
\begin{itemize}
  \item Profile ({\it profile}): Referencia o perfil que definiu apoiou
o comentário;
  \item Comment ({\it comment}): Referencia o comentário classificado;
\end{itemize}

\begin{figure}[h]
\center
\includegraphics[scale=0.5]{support-model.png}
\caption{Modelagem da classe para apoiar um comentário}
\label{fig:support-model}
\end{figure}

A definição dessa classe e seus atributos podem ser vistas na
Figura~\ref{fig:support-model}.

\subsection{Utilização}

\begin{figure}[h]
\center
\includegraphics[scale=0.6]{comment-view-support.png}
\caption{Visualização de um comentário com botão para apoiar o
comentário}
\label{fig:comment-view-support}
\end{figure}

Na visualização de um comentário do Noosfero foi adicionado um botão para
que um usuário apoie um comentário
(Figura~\ref{fig:comment-view-support}). Esse botão será visualizado por
todos os usuários com permissão de visualização do comentário.

\section{Personalização por perfil}

As definições realizadas no escopo do ambiente também poderão ser
personalizadas no contexto de um perfil.

Em cada tipo de perfil do Noosfero será possível habilitar e desabilitar
cada uma das classificações {\it Etiqueta}, {\it Status} e {\it Apoio}.
Além disso, também poderá criar novas etiquetas, novos status e
desabilitar etiquetas e status. Um administrador
apenas de perfil não terá permissão para editar nem remover uma etiqueta
ou status do ambiente, podendo apenas desabilitá-los em seu perfil. Por
padrão, todas as etiquetas e status do ambiente são habilitadas no
contexto do perfil.

Em um perfil específico, ao classificar com uma etiqueta ou status, os
usuários terão disponíveis as opções do ambiente e as opções específicas do
perfil.

\section{Estatísticas das classificações}

Todas as classificações do plugin geram estatísticas que podem ser
visualizadas pelos usuários com permissões específicas na tela de
administração do plugin.

\subsection{Label}

Os avaliadores poderão visualizar quantos comentários receberam cada uma
das etiquetas cadastradas no ambiente.

\subsection{Status}

Os avaliadores poderão visualizar quais status foram definidos para
cada um dos comentários, facilitando a conclusão final sobre o que
deverá ser feito sobre o trecho comentado.

\subsection{Support}

Os usuários com permissão de visualização de comentários poderão
visualizar quais e quantos usuários apoiaram determinado comentário.

Os avaliadores poderão visualizar quais comentários foram mais apoiados,
auxiliando o processo de consulta e participação. Os comentários mais
apoiados podem influenciar o status que o avaliador irá definir para um
comentário.

\section{Exportação dos dados}

Os dados resultantes da consulta pública serão exportados no formato CSV
para que possa ser analisado também fora do ambiente. A sugestão de
formatação pode ser visualizada no Apêndice~\ref{App:PluginCSVFile}.

Esse arquivo gerado lista os trechos, seus comentários e as
justificativas e sugestões de alteração de texto de cada um dos
comentários.

\newpage

\section{Considerações finais}

Neste documento foi apresentada a especificação e modelagem do código
desenvolvido para o Portal de Consulta Pública. As telas incluídas nesse
documento é apenas uma sugestão e pode ser alterada após uma avaliação
de um especialista em {\it Design de interfaces} para que seja
assegurada a qualidade visual e de usabilidade do sistema.

Lembramos que para tornar o Portal de Consulta Pública realmente um canal de
consulta e participação popular na discussão e na definição da agenda
prioritária do país, é necessário que além da proposta de código
incluída nesse documento também haja um esforço fora do sistema.

O incentivo à participação nas consultas deve ser feito através de
mobilizações na rede e fora da rede. Essas mobilizações deverão
incentivar a sociedade a registrar suas ideias no Portal de Consulta
Pública, para que assim ela se torne uma das ferramentas de
sistematização das contribuições da sociedade.

\vspace{1cm}

Sem mais nada a acrescentar, coloco-me à disposição.

\vspace{1cm}

\begin{minipage}{\textwidth}
  Brasília/DF, \DiaEntrega \ de \MesEntrega\\[1cm]
  \textbf{\MyName}\\
  \small Consultora do PNUD
\end{minipage}

\newpage
\appendix
\appendixpage
\section{Página inicial do Portal de Participação do governo}
\label{Att:PaginaInicial}

\begin{figure}[h]
\center
\includegraphics[scale=0.16]{pagina-inicial.png}
\caption{Página inicial do Portal de Participação}
\label{fig:pagina-inicial}
\end{figure}

\newpage
\section{Cabeçalho do Portal de Participação do governo}
\label{Att:CabecalhoPortal}

{\tiny
  \begin{verbatim}
  <div id="barra-brasil">
  </div>
  <div class="header-content">
    <div role="banner" id="header">
      <div>
        <ul id="accessibility">
          <li>
            <a id="link-conteudo" href="#content" accesskey="1">
              Ir para o conteúdo
              <span>1</span>
            </a>
          </li>
          <li>
            <a id="link-navegacao" href="#barra-psocial"
  accesskey="2">
              Ir para o menu
              <span>2</span>
            </a>
          </li>
          <li>
            <a id="link-buscar" href="#portal-searchbox"
  accesskey="3">
              Ir para a Busca
              <span>3</span>
            </a>
          </li>
          <li>
            <a id="link-rodape" href="#theme-footer"
  accesskey="4">
              Ir para o rodapé
              <span>4</span>
            </a>
          </li>
        </ul>
        <ul id="portal-siteactions">
          <li>
            <a href="#">Acessibilidade</a>
          </li>
          <li>
            <a href="#">Alto Contraste</a>
          </li>
          <li>
            <a href="#">Mapa do Site</a>
          </li>
        </ul>
  
        <div id="logo">
          <a title="Participa.br" href="/">
            <span id="portal-title">Participa.br</span>
          </a>
        </div>
  
        <div role="search" id="portal-searchbox">
          <form>
            <input type="text" autocomplete="off"
  name="SearchableText" size="18" title="Buscar no Site"
  placeholder="Buscar no Site" accesskey="3"
  class="searchField" id="searchGadget">
            <input type="submit" class="searchButton"
  value="Buscar"></form>
        </div>
  
        <div id="social-icons">
          <ul>
            <li>
              <a id="sb_face" title="Facebook"
  href="http://www.facebook.com/participa"><span>Facebook</span></a>
            </li>
            <li>
              <a id="sb_tweet" title="Twitter"
  href="http://twitter.com/participa"><span>Twitter</span></a>
            </li>
            <li>
              <a id="sb_youtb" title="Youtube"
  href="http://youtube.com/user/participa"><span>Youtube</span></a>
            </li>
            <li>
              <a id="sb_flickr" title="Flickr"
  href="#"><span>Flickr</span></a>
            </li>
          </ul>
        </div>
      </div>
  
      <div id="sobre">
        <ul>
          <li id="link-faq">
            <a href="/portal/faq">Perguntas frequentes</a>
          </li>
          <li id="link-contact">
            <a href="/contact/portal/new">Contato</a>
          </li>
        </ul>
      </div>
    </div>
  </div>
  
  <div id="barra-psocial">
    <div id="categories_menu">
      <%= render :file => 'categories.rhtml' %>
    </div>
  </div>
  \end{verbatim}
}

\newpage
\section{Código html para gerar as categorias e bolas da página
inicial}
\label{Att:CategoriasBolas}

{\tiny
  \begin{verbatim}
  <div id="destaque-temas">
    <div id="notice-contribua">CONTRIBUA PARA OS GRANDES TEMAS</div>
    <div id="circles">
      <a href="http://participa.gov.br/cat/areas-tematicas/saude"
      class="circle c1" style="top:50px;"><span>Saúde</span></a>
      <a href="#" class="circle c1" style="top:30px;left:270px;"><span
      class="twowords">Reforma Política</span></a>
      <a href="#" class="circle c1" style="top:230px;left:156px;"><span
      class="twowords">Soberania Digital</span></a>
      <a href="#" class="circle c1" style="top:25px;left:600px;"><span
      class="twowords">Mobilidade Urbana</span></a>
      <a href="#" class="circle c1"
      style="top:255px;left:487px;"><span>Reciclagem</span></a>
      <a href="#" class="circle c1"
      style="top:185px;left:762px;"><span>Educação</span></a>
      
      <a href="#" class="circle c2"
      style="top:230px;left:50px;"><span>Finanças</span></a>
      <a href="#" class="circle c2"
      style="top:209px;left:326px;"><span>Gestao</span></a>
      <a href="#" class="circle c2"
      style="top:346px;left:388px;"><span>Cultura</span></a>
      <a href="#" class="circle c2"
      style="top:145px;left:518px;"><span>Esporte</span></a>
      <a href="#" class="circle c2"
      style="left:446px;"><span>Política</span></a>
      
      <a href="#" class="circle c3"
      style="top:88px;left:187px;"><span>Ciencia</span></a>
      <a href="#" class="circle c3"
      style="top:112px;left:856px;"><span>Pecuária</span></a>
      <a href="#" class="circle c3"
      style="top:348px;left:877px;"><span>Pesca</span></a>
      <a href="#" class="circle c3"
      style="top:110px;left:776px;"><span>Engenharia</span></a>
      <a href="#" class="circle c3"
      style="top:335px;left:664px;"><span>Comunicação</span></a>
      <a href="#" class="circle c3"
      style="top:200px;left:687px;"><span>Política</span></a>
      
      <a href="#" class="circle c4"
      style="top:166px;left:172px;"><span>Gestão</span></a>
      <a href="#" class="circle c4"
      style="top:342px;left:744px;"><span>Indígena</span></a>
      <a href="#" class="circle c4"
      style="top:209px;left:617px;"><span>Assistencia</span></a>
      
      <a href="#" class="circle c4"
      style="top:317px;left:332px;"><span>Tecnologia</span></a>
      <a href="#" class="circle c4"
      style="top:277px;left:422px;"><span>Turismo</span></a>
    </div>
  </div>
  \end{verbatim}
}

\newpage
\section{Folha de Estilo - CSS}
\label{Att:CSS}

{\tiny
  \begin{verbatim}
  @import url(../profile-base/style.css);
  
  /* Main style page */
  body {
      background: #FFF;
  }
  
  #link-go-content {
      display: none;
  }
  
  #wrap-1 {
      width: 100%;
      margin: 0px;
  }
  
  #theme-header {
      font-size: 12px;
  }
  
  #wrap-2 {
      max-width: 960px;
      margin: auto;
      border: 0px;
      margin-top: 160px;
  }
  
  #main-content-wrapper-1,
  #main-content-wrapper-2,
  #main-content-wrapper-3,
  #main-content-wrapper-4,
  #main-content-wrapper-5,
  #main-content-wrapper-6,
  #main-content-wrapper-7,
  #main-content-wrapper-8 {
      background-image: none;
      background: none;
  }
  
  /* Overwriting Brasil gov style */
  
  body {
    font-size: 12px;
  }
  
  label {
    font-weight: normal;
  }
  
  #portal-siteactions {
     padding-bottom: 5px;
     margin-top: -10px;
  }
  
  /* Bar Psocial Style */
  
  #user {
     top: -30px;
     font-size: 12px;
  }
  
  #user form {
      display:none;
  }
  
  #barra-brasil {
      box-shadow: 0px 0px 10px #DFDFDF inset;
      z-index: 9999;
      position: relative;
      width: 100%;
  }
  
  #barra-psocial {
      position: relative;
      height: 40px;
      margin: auto;
      background: url(images/barra-psocial-bg.png) repeat-x;
  }
  
  #barra-psocial li {
    float: left;
  }
  
  #assets-menu {
      background: #E8E8E8;
      top: 35px;
      left: 80px;
      min-width: 132px;
  }
  #assets-menu a {
      border: 1px solid #E8E8E8;
  }
  
  #search-header {
      position: relative;
      width: 960px;
      margin: auto;
  }
  #search-header .search-field {
      width: 200px;
      position: absolute;
      top: 100px;
      right: 20px;
      z-index: 999;
  }
  #search-header input.button.with-text {
      width: 22px;
      text-indent: -1000px;
      float: right;
      height: 30px;
      max-height: 30px;
      border: none;
      background-color: transparent;
  }
  #search-header #q {
      width: 170px;
      height: 25px;
      border-radius: 5px;
  }
  
  #categories_menu {
      max-width: 960px;
      margin: auto;
  }
  
  #cat_menu {
      background: url(images/logo-PS-barra.png) no-repeat center
  left;
      height: 40px;
      padding-left: 60px;
  }
  
  #cat_menu li {
      list-style: none;
      font-size: 12px;
      font-weight: bold;
      padding: 15px 20px;
      text-transform: uppercase;
  }
  
  #cat_menu li#category1 {
      color: #ba4a00
  }
  #cat_menu li#category2 {
      color: #3b7390
  }
  #cat_menu li#category3 {
      color: #643c67
  }
  #cat_menu li#category4 {
      color: #826938
  }
  #cat_menu li#category5 {
      color: #1D571F
  }
  #cat_menu li#category6 {
      color: #017b16
  }
  #cat_menu li#category7 {
      color: #1a55dd
  }
  #cat_menu li#category8 {
      color: #753900
  }
  #cat_menu li#category9 {
      color: #56762b
  }
  #cat_menu li#category10 {
      color: #3867b7
  }
  #cat_menu li#category11 {
      color: #00439e
  }
  #cat_menu li#category12 {
      color: #00A0DB
  }
  #cat_menu li#category13 {
      color: #AD6500
  }
  #cat_menu li#category14 {
      color: #DE9200
  }
  
  /* Search Button */
  #search-button a {
      display: inline-block;
      width: 29px;
      height: 25px;
      margin-right: 3px;
      margin-top: 10px;
  }
  
  #search-button a:hover {
      opacity: 0.6;
  }
  
  #search-button #sb_search {
  background-image: url(images/search.png);
  background-size: 100% 100%;
  }
  
  #search-button span { display: none; }
  
  /* Social Buttons */
  #social-buttons a {
      width: 18px;
      height: 20px;
      margin-right: 3px;
      margin-top: 10px;
  }
  
  #social-icons a {
      width: 19px;
      height: 19px;
      display: inline-block;
      background-repeat: no-repeat;
  }
  
  #social-icons a:hover {
      opacity: 0.6;
  }
  
  #social-icons #sb_face {
  background-image: url(images/icone-facebook.png);
  }
  
  #social-icons #sb_tweet {
  background-image: url(images/icone-twitter.png);
  }
  
  #social-icons #sb_youtb {
  background-image: url(images/icone-youtube.png);
  }
  
  #social-icons #sb_flickr {
  background-image: url(images/icone-flickr.png);
  }
  
  #social-icons span { display: none; }
  
  /* Top links */
  #top-links {
    padding-bottom: 3px;
  }
  
  #top-links a {
      display: inline-block;
      font-size: 10px;
      margin-right: 3px;
      margin-top: 15px;
      color: #E7F1E9;
      text-decoration: none;
      text-transform:uppercase;
  }
  
  #top-links a:hover {
      opacity: 0.6;
  }
  
  #top-links #link-faq {
      border-right: 1px solid #E7F1E9;
      padding-right:5px;
  }
  
  /* Title Header */
  
  #content .main-block h1,
  #content .main-block h2,
  #content .main-block h3,
  #content .main-block h4 {
      font-family:  'Open Sans', Arial, Helvetica, sans-serif;
  }
  
  #content .main-block h1,
  #not-found h1,
  #access-denied h1 {
      color: #2C67CD;
      font-size: 1.7em;
      padding: 7px 0;
      margin-left: 0 !important;
      padding-left: 0.3em;
      border-bottom: 1px solid #CCCCCC;
      border-top: 2px solid #172838;
  }
  
  #content .title {
      font-weight: normal; !important;
      padding-right: 70px;
  }
  
  #content .main-block h1 {
      font-size: 2.3em !important;
      font-weight: bold !important;
  }
  
  #content .main-block h2 {
      font-size: 1.8em !important;
      min-height: 48px;
  }
  
  #content .main-block h3 {
      font-size: 1.5em !important;
      min-height: 48px;
  }
  
  #content .main-block h4 {
      font-size: 1.3em !important;
      min-height: 48px;
  }
  
  div#article-parent {
      position: absolute;
      right: 0px;
      top: 0px;
  }
  
  /* Menu List left */
  
  #content .box-2 .block-title {
      font-size: 12px;
      text-align: left;
      border-top: 4px solid #757575;
      background: #eeefff;
      border-bottom: none;
      color: #757575;
      padding: 8px 8px 24px 10px;
      text-transform: uppercase;
      margin: 0;
  }
  
  #content .box-2 .link-list-block li a.link-this-page {
      width: auto;
      border-right: none;
      font-weight: bold;
      color: #436976 !important;
      border-top: 2px solid #64946e !important;
      border-bottom: 2px solid #64946e !important;
      background-color: #eeefff;
      border-radius: 0px;
  }
  
  #content .box-2 .link-list-block li a {
      font-size: 14px;
      line-height: 1em;
      color: #436976;
      background-color: #FFF;
      border-radius: none;
      padding: 0.6em 1.5em;
  }
  
  #content .box-2 .link-list-block li a:hover {
      background-color: #FFF;
      color: #000;
  }
  
  #content .box-2 .link-list-block li {
      border-bottom: 1px solid #ddd;
      border-top: none;
      padding: 0;
      margin: 0;
  }
  
  /* Blocks main on home */
  
  .action-home-index .main-block {
      display: none;
  }
  
  .action-home-index .box-2 {
      display: none;
  }
  
  .action-home-index .box-1 {
      margin-left: 0px;
  }
  
  /* Blocks main style */
  
  .action-home-index #content .box-1 .block-title {
      padding: 0.6em 1.5em;
      color: #444;
      text-transform: uppercase;
      font-variant: normal;
  }
  
  .action-home-index #content .box-1 .main-block {
      background: #FFF;
      width: 100%;
      clear: both;
  }
  
  .action-home-index #content .box-1 .article-block {
      background: none;
  }
  
  #content .blog-post .title {
      text-align: left;
  }
  
  /* Box styles */
  .box-1 {
      Xmargin: 0 0 0 170px;
  }
  
  .action-home-index .box-2 {
      width: 160px;
  }
  
  .action-home-index #content .box-3 .block {
      width: 210px;
      margin: 0px 0px 0px 30px;
      float: left;
  }
  
  /* Editorial Area */
  
  #content .box-1 .news-area {
      width: 32%;
      margin-right: 10px;
  }
  
  #content .box-1 a {
      text-decoration: none;
  }
  
  #content .box-1 .news-area h3 {
      background: #DFDFDF;
      text-decoration: none;
      line-height: 40px;
      height: 40px;
      min-height: 40px;
      border-top: 8px solid #545454;
      padding-left: 10px;
      text-transform: uppercase;
      font-weight: normal;
      font-size: 20px;
  }
  
  #content .box-1 .news-area a.news-see-more {
      position: relative;
      border-top: 3px solid #545454;
      background-color: #DFDFDF;
      width: auto;
      display: block;
      text-align: right;
      padding: 5px 15px;
  }
  
  #content .box-1 .news-area ul {
      background: #FFF;
      border: none;
      background-image: none;
  }
  
  .blog-post {
      background: none !important;
      margin: 0px;
  }
  
  /* Block Display Content style */
  
  .box-1 .display-content-block {
      font-size: 14px;
  }
  
  .box-1 .display-content-block h2 {
      font-size: 24px;
      font-weight: bold;
  }
  
  /* Block Article style */
  
  .action-home-index #content .box-1 .article-block {
      width: 100%;
      font-size: 14px;
  }
  
  .action-home-index #content .box-1 .article-block .block-title {
      display: none;
  }
  
  .action-home-index #content .box-1 .article-block h2 {
      font-size: 34px;
      font-weight: bold;
  }
  
  .action-home-index #content .box-1 .article-block .read-more {
      float: right;
  }
  
  .action-home-index #content .box-1 .article-block .read-more a {
      display: block;
      height: 39px;
      width: 161px;
      text-align: center;
      padding: 7px;
      background: url(images/button-read-more2.png) no-repeat;
  }
  
  .action-home-index #content .box-1 .article-block .read-more a,
  .action-home-index #content .box-1 .article-block .read-more a:hover {
     color: transparent !important;
  }
  
  .action-home-index #content .box-1 .article-block .read-more
  a:first-child {
      display: none;
  }
  
  /* Block My Network style */
  
  .action-home-index #content .my-network-block {
      background: #eeefff;
  }
  
  .action-home-index #content .my-network-block .block-title {
      background: #757575;
      padding: 0.6em 1.5em;
      color: #FFF;
  }
  
  .action-home-index #content .my-network-block ul {
      padding: 10px 0px 10px 20px;
  }
  
  .action-home-index #content .my-network-block .my-network-actions
  a.button {
      width: 100px;
      display: block;
      margin-bottom: 5px;
  }
  
  /* Blocks profiles and enterprises */
  .recent-documents-block ul {
      padding: 0px 0px 0px 30px;
  }
  
  .box-1 .menu-submenu {
      bottom: 80px;
      right: -40px;
  }
  
  .box-1 .common-profile-list-block .vcard {
      border-radius: 0px;
      -moz-border-radius: 0px;
  }
  
  .box-1 .common-profile-list-block .profile_link {
      width: 65px;
      height: 80px;
  }
  
  .box-1 .common-profile-list-block span {
   width: auto;
  }
  
  .box-1 .common-profile-list-block .profile-image {
      display: block;
      height: auto;
      width: auto;
      text-align: center;
      margin: 0px;
      padding: 0px;
      padding-bottom: 5px;
  }
  
  .box-1 .common-profile-list-block {
      margin-left: 15px;
  }
  
  .action-home-index #content .communities-block .block-title,
  .action-home-index #content .display-people-block .block-title {
      display: none;
  }
  
  #content .display-people-block .banner-span {
      background-color: #f15921;
      color: #FFF;
      font-size: 24px;
      font-weight: normal;
  }
  
  .box-1 .display-people-block .banner-span,
  .common-profile-list-block .vcard a {
      width: 108px;
      height: 57px;
  }
  
  #content .display-people-block .vcard a.profile_link {
      height: 52px;
      margin: 0;
      width: 52px;
  }
  
  .box-1 .block-footer-content {
      clear: both;
      text-align: right;
      font-size: 11px;
      color: #000;
      position: relative;
      padding: 5px;
  }
  
  #content .box-1 .block-footer-content a {
      position: relative;
  }
  
  /* Blocks tags */
  .action-home-index #theme-footer .block.tags-block {
      background: #034811;
      min-height: 50px;
      padding: 20px 0;
      width: 100%;
      margin-bottom: 0px;
      background-image: url(images/bg-fundo-verde-tags.png);
  }
  
  .action-home-index #theme-footer .block.tags-block .block-inner-1 {
      max-width: 960px;
      margin: auto;
  }
  .action-home-index #theme-footer .block.tags-block .block-title {
      display: none;
  }
  
  .action-home-index #theme-footer .tags-block .block-footer-content a {
      color: #FFFFFF;
  }
  
  .action-home-index #theme-footer .block.tags-block a,
  .action-home-index #theme-footer .block.tags-block a:visited,
  .action-home-index #theme-footer .block.tags-block a:hover {
      color: #E7F1E9;
  }
  
  .action-home-index #theme-footer .block.tags-block a:hover {
      text-decoration: underline;
  }
  
  /* Menu List footer */
  
  #content .box-3 .block-title {
      font-size: 12px;
      text-align: left;
      border-top: none;
      background: #FFF;
      border-bottom: none;
      color: #757575;
      padding: 5px;
      text-transform: uppercase;
      margin: 0;
  }
  
  #content .box-3 .link-list-block li a.link-this-page {
      width: auto;
      border-right: none;
      font-weight: bold;
      background-color: #eeefff;
      border-radius: 0px;
  }
  
  #content .box-3 .link-list-block li a {
      font-size: 14px;
      line-height: 1em;
      color: #545454;
      background-color: #FFF;
      border-radius: none;
      padding: 0.6em 1.5em;
  }
  
  #content .box-3 .link-list-block li a:hover {
      background-color: #FFF;
      color: #436976;
  }
  
  #content .box-3 .link-list-block li {
      border-bottom: none;
      border-top: none;
      padding: 0;
      margin: 0;
  }
  
  .agenda-item a {
    color: black;
  }
  .agenda-item a:visited {
    color: black;
    font-weight: normal;
  }
  .agenda-item a:hover {
    color: black;
    text-decoration: underline !important;
  }
  
  #box-organizer .block-target { clear: both; }
  
  /* Custom Icons */
  
  .box-2 .link-list-block ul li .icon-ok,
  .box-2 .link-list-block ul li .icon-eyes,
  .box-2 .link-list-block ul li .icon-edit,
  .box-2 .link-list-block ul li .icon-photos {
      background-position: 0px -325px !important;
      height: 27px;
      padding: 13px 0px 0px 45px !important;
      margin-top: 5px;
      margin-bottom: 5px;
  }
  
  .box-2 .link-list-block ul li .icon-ok {
      background-position: 0px -165px !important;
  }
  .box-2 .link-list-block ul li .icon-eyes {
      background-position: 0px -125px !important;
  }
  .box-2 .link-list-block ul li .icon-edit {
      background-position: 0px -245px !important;
  }
  
  /* Footer */
  
  #theme-footer {
      width: 100%;
  }
  
  #footer-logos {
      background: #00420C;
      max-width: 100%;
      padding: 2em 0;
      height: 49px;
  }
  
  #footer-logos div,
  #footer-license {
      max-width: 960px;
      margin: 0 auto;
  }
  
  #footer-logos a {
    display: block;
    height: 49px;
    float: left;
  }
  
  #footer-logos span {
    display: none;
  }
  
  #footer-logos .logo-acesso {
    background: transparent url(images/acesso-a-informacao.png) center
  center no-repeat;
    width: 107px;
  }
  
  #footer-logos .logo-brasil {
    background: transparent url(images/brasil.png) center center
  no-repeat;
    width: 153px;
  }
  
  #footer-logos .logo-sgpr {
    background: transparent url(images/sgpr.png) center center no-repeat;
    width: 187px;
    margin-right: 30px;
  }
  
  #footer-logos .institucionais {
    float: right;
  }
  
  #footer-license {
    text-align: left;
  }
  
  /* padrao da barra verde - estatistica */
  .action-home-index #content .box-1 .statistics-block {
     width: 100%;
     display: inline-block;
     margin: 0px;
     padding: 28px 0px 15px 0px;
  }
  
  .action-home-index .statistics-block .block-title {
     display: none;
  }
  
  .action-home-index .statistics-block-data ul{
          list-style:none;
  }
  
  .action-home-index .statistics-block-data ul li{
          float:left;
          width:146px;
          margin:0 20px 20px 0px;
          text-align:center;
          color:#fff;
  }
  
  .action-home-index .statistics-block-data ul li .amount{
          font-size:40px !important;
          font-weight:bold;
          width:136px;
  }
  
  .action-home-index .statistics-block-data ul li .label{
          font-size:14px !important;
          clear:both;
          float:left;
          width:136px;
          color:#98baa5;
  }
  
  #header input.searchButton {
    background-image: url(images/search-button.gif);
  }
  
  /* HOME */
  
  .action-home-index #wrap-2 {
    max-width: none;
    padding: 0px;
  }
  
  .action-home-index #boxes {
    margin-top: 0px;
  }
  
  .action-home-index #content {
    margin: 0px;
  }
  
  .action-home-index .home-blocks-inner {
    max-width: 960px;
    margin: auto;
  }
  
  .action-home-index #content-blocks {
    margin-top: 30px;
    clear: both;
  }
  
  .action-home-index #content-blocks .home-blocks-inner {
    text-align: center;
  }
  
  .action-home-index #content .box-1 .display-content-block {
      width: 300px;
      display: inline-block;
      text-align: left;
  }
  
  .action-home-index #content #content-blocks-inner {
      max-width: 960px;
  }
  
  #content .box-1 #content .statistics-block li,
  #content .box-1 #content .display-content-block li {
      list-style: none;
      margin: 0px;
  }
  
  .action-home-index #content .communities-block,
  .action-home-index #content .display-people-block {
      display: inline-block;
      width: 50%;
      vertical-align: top;
      margin-top: 30px;
  }
  
  .action-home-index #content .box-1 .block {
      margin-right: 0px;
  }
  
  .action-home-index #statistics-blocks {
          background:#0c763e;
  }
  
  .action-home-index .statistics-block {
          padding:10px 0 25px 0;
  }
  
  .action-home-index #circles-blocks {
      width: 100%;
      background: url(images/bg-palacio-do-planalto.jpg);
      padding-top: 20px;
  }
  
  .action-home-index #block.track-list-blocks {
      width: 100%;
      background: url(images/bg-pessoas.jpg);
  }
  
  .action-home-index #profile-blocks {
      background: url(images/bg-linhas-cinza.png);
  }
  
  \end{verbatim}
}


\end{document}
