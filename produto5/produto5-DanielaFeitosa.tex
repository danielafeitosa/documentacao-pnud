\documentclass[[a4paper,11pt]{article}
\renewcommand{\rmdefault}{ptm}
\usepackage[scaled=0.92]{helvet}
\usepackage{courier,xcolor,colortbl,listings,parskip,graphicx,fancyvrb,fancyhdr,lastpage}
\usepackage{float,framed}
\normalfont
\usepackage[T1]{fontenc}
\setlength{\parskip}{7pt}
\usepackage[toc,page]{appendix}
\usepackage[hmargin=2.5cm,vmargin=2cm]{geometry}
\usepackage[utf8]{inputenc}
\usepackage[brazil]{babel}
\pagestyle{fancy}
\setlength{\headheight}{120pt}
\setlength{\headsep}{30pt}
\setlength{\textheight}{550pt}
\renewcommand{\headrulewidth}{0pt}
\lhead{}
\rhead{}
\chead{\includegraphics{brasao.jpg}\\
        \large \textbf{PRESIDÊNCIA DA REPÚBLICA}\\
        \large SECRETARIA-GERAL\\
        \large Secretaria-Executiva}
\cfoot{}
\rfoot{\thepage /\pageref{LastPage}}
\hyphenation{par-ti-ci-pa-ção}
\bibliographystyle{ieeetr}

\newcommand{\MyName}{Daniela Soares Feitosa}
\newcommand{\MyEmail}{daniela@colivre.coop.br}
\newcommand{\ContractNumber}{2013/000292}
\newcommand{\ProjectCode}{Projeto PNUD BRA/12/018}
\newcommand{\NomeSecretaria}{Secretaria Geral da Presidência da República}
\newcommand{\SiglaSecretaria}{SG/PR}
\newcommand{\ProductNumber}{05}
\newcommand{\ProductDescription}{Documento com proposta para 
desenvolvimento do código do tema das
comunidades contendo exemplos e códigos.}
\newcommand{\MesEntrega}{Maio de 2014}
\newcommand{\DiaEntrega}{26}

\begin{document}
\lstset{language=Ruby}
\definecolor{light-gray}{gray}{0.95}
\lstdefinestyle{codeFrame}{backgroundcolor=\color{light-gray},frame=lines}

\textbf{\ProjectCode \ -} \ProductDescription

\vspace{3cm}

\begin{minipage}{0.5\textwidth}
  \textbf{Consultora: \MyName}
  \newline
  \textbf{Contrato nº: \ContractNumber}
  \newline
  \textbf{Produto / nº: \ProductNumber}
\end{minipage}

\vspace{2cm}

\textbf{Assinatura do Consultor}

\begin{framed}
Local e data: Brasília/DF, \DiaEntrega \ de \MesEntrega 
\newline
\newline
Assinatura do Consultor: \line(1,0){300}
\end{framed}

\vspace{1cm}

\textbf{Assinatura do Supervisor}

\begin{framed}
Atesto que os serviços foram prestados conforme estabelecido no Contrato
de Consultoria.
\newline
\newline
Local e data: Brasília/DF, \DiaEntrega \ de \MesEntrega 
\newline
\newline
Assinatura e Carimbo: \line(1,0){300}
\end{framed}

\clearpage
\newcolumntype{g}{>{\columncolor{light-gray}}l}

\begin{center}
  \begin{tabular}{| g | p{10cm} |}
    \hline
    \textbf{Título} & \ProductDescription \\ \hline
    \textbf{Língua do documento} & Português - Brasil \\ \hline
    \textbf{Documentação de referência} & Português \\ \hline
    \textbf{Unidade responsável} & \NomeSecretaria \
(\SiglaSecretaria) \\ \hline
    \textbf{Criador} & \MyName - \MyEmail \\ \hline
    \textbf{Taxonomias} & Desenvolvimento \\ \hline
    \textbf{Data de aprovação} &  \\ \hline
    \textbf{Público} & \SiglaSecretaria, Parceiros e Sociedade
Civil \\ \hline
    \textbf{Faz parte do} & \ProjectCode \\ \hline
    \textbf{Em conformidade com a} & \NomeSecretaria \\ \hline
    \textbf{Documentos anexos} & Página da Comunidade OSC - Organizações da
Sociedade Civil; CSS do cabeçalho do tema de comunidade; Código javascript para
incluir uma classe identificadora; Código CSS para estilizar os blocos com o padrão
colorido \\ \hline
    \textbf{Revisado em} &  \\ \hline
  \end{tabular}
\end{center}

\clearpage

\tableofcontents
\clearpage
\listoffigures

\clearpage

\section{Introdução}

A participação social no Brasil representa princípio
jurídico-institucional
presente na Constituição Federal de 1988, que a definiu como forma de
afirmação da
democracia e da consolidação da cidadania. Ao incorporar esse princípio
como
referência para a gestão pública, o Governo Federal aprimora os
processos de interação
do Estado com a sociedade e cria as condições institucionais para a
prática da
democracia participativa. Com isso, verifica-se que, além da crescente
participação
social nas decisões governamentais, as políticas públicas ganham maior
legitimidade,
uma vez que expressam as atuais condições socioeconômicas e culturais da
população
brasileira em suas diferentes realidades regionais.

Na estrutura administrativa do Poder Executivo Federal, cabe à
Secretaria-Geral
da Presidência da República (SG/PR) a função de intermediar as relações
do Governo
com as entidades da sociedade civil, conforme competências definidas
pela Lei
10.683/2003 e pelo Decreto no. 7.688/2012. Assim, a SG/PR é órgão
incumbido de
assessorar diretamente a Presidenta da República e os órgãos e entidades
do Governo
Federal no relacionamento e na articulação com os movimentos sociais, o
que inclui a
criação e a implementação de canais que assegurem a consulta e a
participação popular
na discussão e na definição da agenda prioritária do país.

O Brasil tem um rico histórico de efetivação da democracia
participativa, sendo
reconhecido mundialmente. Os instrumentos institucionalizados como
conselhos de
políticas públicas e conferências nacionais foram profundamente
ampliados na última
década, contando com um legado volumoso de práticas e realizações. A
maioria dos
programas de governo já conta com participação social prevista em pelo
menos uma de
suas etapas. As práticas trazidas pelas novas mídias e pela cultura
digital podem
interagir nesses espaços fortalecendo, ampliando e aprofundando a
democracia
participativa, já na abordagem do novo século, pós redes sociais
digitais.

Incentivando os atores a conectar perfis, blogs e demais instâncias de
produção
de conteúdo na rede, o Portal de Participação Social poderá se
estabelecer como um
repositório agregador do conhecimento sobre participação social hoje
disperso na
internet e nas instâncias governamentais. A partir de uma interface
clara que possibilite
a navegação pelos temas, o Portal de Participação Social pode ser um
espaço onde os
diferentes agentes de governo, movimentos, organizações e cidadãos em
geral
encontrarão solo fértil e facilitado para o exercício da pesquisa e do
diálogo.

Tendo por base a premissa de que a incursão e abertura de canais de
acesso ao
poder público na rede aumenta o conhecimento das ações do governo e
diminui as
barreiras para participação de cidadãos, entidades e movimentos, esse
projeto visa a
construção de um conjunto de ferramentas que poderão ser utilizadas por
gestores e
servidores para proporcionar novas formas de participação a serem
apropriadas pela
cidadania. Além disso, esse portal também buscará dar evidência às
formas de
participação existentes no sentido de contextualizar, organizar e
facilitar o acesso do
cidadão às formas de incidir nas diversas etapas das políticas públicas
do governo
brasileiro.

\section{Apresentação}

Este documento é parte da consultoria para o projeto
\textbf{Desenvolvimento de Metodologias de Articulação e Gestão de
Políticas Públicas para Promoção da Democracia Participativa}
(BRA/12/018), firmado entre a Secretaria-Geral da Presidência da República
(SG/PR) e o Programa das Nações Unidas para o Desenvolvimento (PNUD).
A consultoria tem como objetivo a especificação da construção dos códigos das
metodologias de organização da informação e interação participativa do
portal da participação social.

O presente documento apresenta a proposta para
desenvolvimento do código do tema das comunidades contendo exemplos e códigos.
Essa proposta está configurada como produto 5 da consultoria técnica.

Neste documento será apresentada a especificação e
modelagem para desenvolvimento do código necessário para aplicação do
tema de comunidades criadas no portal Participa.br como definido 
pela equipe do projeto Participa.br em parceria com a equipe da
Secretaria-Geral da Presidência da República responsável 
pela agenda do Marco Regulatório das Organizações da Sociedade Civil (MROSC).

O Portal de Consulta Pública utiliza o software livre Noosfero,
plataforma web para redes sociais, então o código produzido deverá ser público
e divulgado para a comunidade e para os que desenvolvem e se utilizam do
software.

\section{Personalização de comunidades no Participa.br}

No Noosfero, plataforma base do Participa.br, o foco é rede social para
produção e publicação de conteúdo na Internet. Cada perfil criado na plataforma
tem também a função de site e funciona como uma página personalizável.

Os perfis do tipo "Comunidade" criados no Participa.br servirão como um 
espaço onde os diferentes agentes de governo, movimentos, organizações e 
cidadãos em geral se organizarão para compartilhar interesses, preferências
e, utilizando o conjunto de ferramentas disponíveis, proporcionar novas 
formas de participação a serem apropriadas pela cidadania.

Criada por uma pessoa e mantida por uma ou mais, o perfil "Comunidade" é 
caracterizado por um coletivo de usuários que tem como objetivo agregar pessoas
em torno de interesses comuns. 

A personalização de uma Comunidade no Participa.br envolve quatro itens principais:

\subsection{Alteração de template}

Acessando a seção "Editar Aparência" pelo painel de controle da comunidade 
(Figura~\ref{fig:editar-template}), o administrador da comunidade poderá escolher
como será a organização de colunas da página.

As imagens nessa tela tentam mostrar como ficará a organização das colunas da
página. O retângulo mais largo representa a área com o conteúdo principal da
página e os outros retângulos representam as colunas para inserção de blocos.

Todas as opções disponíveis inclui o bloco que irá mostrar o conteúdo principal
da página. A escolha do administrador nessa seção é sobre a presença, quantidade
e posição das colunas para blocos.

\begin{figure}[h]
\center
\includegraphics[scale=0.5]{editar-template.png}
\caption{Tela para escolha de template}
\label{fig:editar-template}
\end{figure}

\subsection{Organização de blocos}

Na seção "Editar blocos laterais" do painel de controle da comunidade 
(Figura~\ref{fig:editar-blocos}), o administrador visualizará as colunas da
página de acordo com o template configurado.

Clicando no botão "Adicionar blocos" o administrador poderá escolher quais
blocos serão inseridos na página. Cada bloco tem uma funcionalidade específica e
deve ser escolhido de acordo com o objetivo da comunidade.

\begin{figure}[h]
\center
\includegraphics[scale=0.5]{editar-blocos.png}
\caption{Tela para adicionar blocos}
\label{fig:editar-blocos}
\end{figure}

\subsection{Edição de cabeçalho e rodapé}

Na seção "Editar Cabeçalho e Rodapé", o administrador pode alterar o que será
visualizado pelos usuários no topo e na parte inferior da sua comunidade. O
cabeçalho e rodapé é visível em todas as páginas da comunidade, portanto, deve
dar preferência por inserir conteúdos estratégicos. No cabeçalho, é
importante que seja inserido um banner sobre o objetivo da comunidade e no
rodapé, informações importantes e que devem ficar sempre visíveis.

\begin{figure}[h]
\center
\includegraphics[scale=0.5]{editar-cabecalho-rodape.png}
\caption{Tela para edição de cabeçalho e rodapé}
\label{fig:editar-cabecalho-rodape}
\end{figure}

\subsection{Alteração do tema}

Também na seção "Editar Aparência" pelo painel de controle da comunidade 
(Figura~\ref{fig:editar-tema}), o administrador da comunidade poderá escolher
alguns temas que ficarão disponíveis para as comunidades.
Além dos temas disponíveis por padrão no Participa.br, é possível criar novos
temas para personalizar as comunidades.

\begin{figure}[h]
\center
\includegraphics[scale=0.5]{editar-tema.png}
\caption{Tela para escolha de tema}
\label{fig:editar-tema}
\end{figure}

\section{Proposta para desenvolvimento do código do tema das 
comunidades contendo exemplos e códigos.}

Neste documento, é proposta uma nova opção de tema de comunidade que foi
utilizada na comunidade OSC (http://www.participa.br/osc) e pode ser aplicada em
outras comunidades no Participa.br.

As ideias que serviram como base para a definição da arquitetura de
informação e do tema de comunidade do portal de Consulta Pública foram
discutidas em reuniões com integrantes da Secretaria Geral da
Presidência da República (SG/PR) em parceria com os responsáveis pela agenda do
Marco Regulatório das Organizações da Sociedade Civil (MROSC).

O gráfico do layout para o tema foi desenvolvido por consultores vinculados à
equipe da agenda do MROSC e aprovado pela equipe gestora do projeto. O tema padrão
pode ser visualizado no
Anexo~\ref{Att:PaginaComunidade}.

Para a codificação do tema foi utilizado como base o tema "Conference". Esse
tema foi produto de uma consultoria que integra o processo de preparação da
Conferencia Nacional sobre Migrações e Refúgio (COMIGRAR).
O tema "Conference" foi implementado por Valéssio Soares de Brito em parceria
com o Laboratório Avançado de Produção, Pesquisa e Inovação em Software da UNB -
Gama, coordenado pelo professor Dr. Paulo Roberto Miranda Meirelles. O tema está
disponível no repositório oficial do projeto: 
https://gitlab.com/participa/conference-theme .

O ideal é que o Participa.br/Noosfero tenha uma estrutura para que os próprios
usuários possam criar e aplicar os temas nas suas comunidades. Enquanto isso não
é desenvolvido, todos os temas do Participa.br devem ser públicos e disponíveis
no repositório oficial do projeto (https://gitlab.com/participa/). A inclusão
dos temas no repositório facilita a manutenção e evolução por outras pessoas, já
que garante a possibilidade de contribuição e o controle de versão dos arquivos.

\subsection{Cabeçalho da comunidade com a identidade visual do Portal}

\begin{figure}[h]
\center
\includegraphics[scale=0.35]{cabecalho-reduzido.png}
\caption{Cabeçalho do portal reduzido}
\label{fig:cabecalho-reduzido}
\end{figure}

\begin{figure}[h]
\center
\includegraphics[scale=0.35]{cabecalho.png}
\caption{Cabeçalho do portal}
\label{fig:cabecalho}
\end{figure}

O cabeçalho do Portal de Participação Social do Governo Federal na internet
segue o modelo utilizados em outros sites do Governo Federal.

À esquerda constam os links para acesso rápido à algumas
seções da página e o nome do Portal.
À direita constam os links de acessibilidade, mapa do site, campo de
busca, links para outras redes sociais e links de perguntas e contato do
Portal.

Para manter a identidade visual do Portal nas comunidades, todas as comunidades
irão apresentar o cabeçalho padrão. Ao acessar uma comunidade, o usuário verá
ema versão reduzida do cabeçalho para permitir um espaço maior para o conteúdo
criado na comunidade (Figura~\ref{fig:cabecalho-reduzido}). Quando o usuário
passar o mouse sobre a barra superior, o cabeçalho se expandirá, mostrando o
cabeçalho completo (Figura~\ref{fig:cabecalho}).  

O código CSS que mostra o cabeçalho completo ao passar o mouse sobre a barra do Portal pode ser visualizado no
Anexo~\ref{Att:CabecalhoComunidade}.

\subsection{Cabeçalho e rodapé da comunidade}

\begin{figure}[h]
\center
\includegraphics[scale=0.5]{banner-destaque.png}
\caption{Banner de destaque da comunidade}
\label{fig:banner-destaque}
\end{figure}

\begin{figure}[h]
\center
\includegraphics[scale=0.5]{logos-apoiadores.png}
\caption{Banner com logomarcas dos apoiadores}
\label{fig:logos-apoiadores}
\end{figure}

O banner no cabeçalho (Figura~\ref{fig:banner-destaque}) e rodapé
(Figura~\ref{fig:logos-apoiadores}) da comunidade são compostos por imagens que
possuem dimensões: 960x200px e 960x125px respectivamente, com imagens
no formato JPG ou PNG, contendo a identidade visual e logo da comunidade no
cabeçalho e as logos de parceiros e realizadores no rodapé.

As imagens utilizadas como exemplos foram utilizadas na comunidade OSC
(http://www.participa.br/osc). Ao utilizar o tema em outra comunidade, as
imagens deverão ser alteradas para se adequar ao conteúdo das comunidades,
respeitando as dimensões informadas acima. 

\subsection{Bloco de Links}

\begin{figure}[h]
\center
\includegraphics[scale=0.4]{bloco-links.png}
\caption{Bloco de links}
\label{fig:bloco-links}
\end{figure}

O Noosfero já possui um bloco de links onde podem ser inseridos quaisquer links
que o administrador da comunidade desejar. Para esse tema, o bloco de links
(Figura~\ref{fig:bloco-links}) foi configurado como um menu com os principais
conteúdos da comunidade.

Seguindo o layout aprovado pela equipe, apenas o primeiro bloco de links da comunidade teria a
apresentação alterada em relação ao tema base "Conference". Para isso, foi
necessária a inclusão de um código javascript
(Anexo~\ref{Att:CodigoJsLinkColoridos}) para incluir uma classe "colored-links" ao redor
do bloco considerado como menu da comunidade. No arquivo CSS, as cores de cada
item do menu foram definidas como apresentado no
Anexo~\ref{Att:CodigoCSSLinkColoridos}.

Alguns itens do menu deveriam ter seu texto alterado quando 
o mouse passasse sobre ele (Figura~\ref{fig:bloco-links-hover}). Para isso, foi desenvolvido um código com javascript
que altera o texto a partir de uma lista dada e, nos casos que o texto tem a
largura maior que o texto original, o espaçamento das letras é reduzido para
evitar que o texto seja mostrado em mais de uma linha e cause estranhamento
visual.

\begin{figure}[h]
\center
\includegraphics[scale=0.4]{bloco-links-hover.png}
\caption{Item do bloco de links com hover}
\label{fig:bloco-links-hover}
\end{figure}

\subsection{Menu para edição da comunidade}

\begin{figure}[h]
\center
\includegraphics[scale=0.5]{menu-rapido.png}
\caption{Menu com atalhos para a administração da comunidade}
\label{fig:menu-rapido}
\end{figure}

Para facilitar a gestão do conteúdo e apresentação da comunidade, foi inserido
um menu com atalho para as principais seções de administração
(Anexo~\ref{Att:MenuAdministracao}). Esse menu só é
apresentado para os usuários logados e com papel de administrador na comunidade.

\subsection{Folha de Estilo - CSS}

O CSS é utilizada para definir a apresentação de documentos escritos em
uma linguagem de marcação, como HTML ou XML. Seu principal benefício é
prover a separação entre o formato e o conteúdo de um documento.

Ao invés de inserir a formatação diretamente no documento, é inserido
um link na página para o arquivo que contém as definições de estilo.

O CSS do tema pode ser visualizada no repositório oficial do Portal
( https://gitlab.com/participa/mrosc-theme/blob/master/style.css ). 

Como os temas do Portal estarão sempre em evolução, o arquivo com o CSS da
comunidade poderá ser atualizado para que os novos elementos
inseridos nas páginas sigam o padrão do tema.

\newpage

\section{Considerações finais}

Neste documento foi apresentada proposta do desenvolvimento do código do tema
das comunidades contendo exemplos e códigos.

O tema já está sendo utilizado numa comunidade ativa do Participa.br, a
comunidade OSC ( http://www.participa.br/osc ) e pode ser utilizada em outras
comunidades.

Com o objetivo de melhorar a usabilidade, apresentação ou atender às
necessidades das outras comunidades que a utilizarem, esse
tema poderá sofrer alterações.

Esse e todos os outros temas desenvolvidos para o Participa.Br ficarão
disponíveis no repositório oficial do projeto ( https://gitlab.com/participa/ ).

\vspace{1cm}

Sem mais nada a acrescentar, coloco-me à disposição.

\vspace{1cm}

\begin{minipage}{\textwidth}
  Brasília/DF, \DiaEntrega \ de \MesEntrega\\[1cm]
  \textbf{\MyName}\\
  \small Consultora do PNUD
\end{minipage}

\newpage
\appendix
\appendixpage
\section{Página inicial do Portal de Participação do governo}
\label{Att:PaginaInicial}

\begin{figure}[h]
\center
\includegraphics[scale=0.16]{pagina-inicial.png}
\caption{Página inicial do Portal de Participação}
\label{fig:pagina-inicial}
\end{figure}

\newpage
\section{Cabeçalho do Portal de Participação do governo}
\label{Att:CabecalhoPortal}

{\tiny
  \begin{verbatim}
  <div id="barra-brasil">
  </div>
  <div class="header-content">
    <div role="banner" id="header">
      <div>
        <ul id="accessibility">
          <li>
            <a id="link-conteudo" href="#content" accesskey="1">
              Ir para o conteúdo
              <span>1</span>
            </a>
          </li>
          <li>
            <a id="link-navegacao" href="#barra-psocial"
  accesskey="2">
              Ir para o menu
              <span>2</span>
            </a>
          </li>
          <li>
            <a id="link-buscar" href="#portal-searchbox"
  accesskey="3">
              Ir para a Busca
              <span>3</span>
            </a>
          </li>
          <li>
            <a id="link-rodape" href="#theme-footer"
  accesskey="4">
              Ir para o rodapé
              <span>4</span>
            </a>
          </li>
        </ul>
        <ul id="portal-siteactions">
          <li>
            <a href="#">Acessibilidade</a>
          </li>
          <li>
            <a href="#">Alto Contraste</a>
          </li>
          <li>
            <a href="#">Mapa do Site</a>
          </li>
        </ul>
  
        <div id="logo">
          <a title="Participa.br" href="/">
            <span id="portal-title">Participa.br</span>
          </a>
        </div>
  
        <div role="search" id="portal-searchbox">
          <form>
            <input type="text" autocomplete="off"
  name="SearchableText" size="18" title="Buscar no Site"
  placeholder="Buscar no Site" accesskey="3"
  class="searchField" id="searchGadget">
            <input type="submit" class="searchButton"
  value="Buscar"></form>
        </div>
  
        <div id="social-icons">
          <ul>
            <li>
              <a id="sb_face" title="Facebook"
  href="http://www.facebook.com/participa"><span>Facebook</span></a>
            </li>
            <li>
              <a id="sb_tweet" title="Twitter"
  href="http://twitter.com/participa"><span>Twitter</span></a>
            </li>
            <li>
              <a id="sb_youtb" title="Youtube"
  href="http://youtube.com/user/participa"><span>Youtube</span></a>
            </li>
            <li>
              <a id="sb_flickr" title="Flickr"
  href="#"><span>Flickr</span></a>
            </li>
          </ul>
        </div>
      </div>
  
      <div id="sobre">
        <ul>
          <li id="link-faq">
            <a href="/portal/faq">Perguntas frequentes</a>
          </li>
          <li id="link-contact">
            <a href="/contact/portal/new">Contato</a>
          </li>
        </ul>
      </div>
    </div>
  </div>
  
  <div id="barra-psocial">
    <div id="categories_menu">
      <%= render :file => 'categories.rhtml' %>
    </div>
  </div>
  \end{verbatim}
}

\newpage
\section{Código html para gerar as categorias e bolas da página
inicial}
\label{Att:CategoriasBolas}

{\tiny
  \begin{verbatim}
  <div id="destaque-temas">
    <div id="notice-contribua">CONTRIBUA PARA OS GRANDES TEMAS</div>
    <div id="circles">
      <a href="http://participa.gov.br/cat/areas-tematicas/saude"
      class="circle c1" style="top:50px;"><span>Saúde</span></a>
      <a href="#" class="circle c1" style="top:30px;left:270px;"><span
      class="twowords">Reforma Política</span></a>
      <a href="#" class="circle c1" style="top:230px;left:156px;"><span
      class="twowords">Soberania Digital</span></a>
      <a href="#" class="circle c1" style="top:25px;left:600px;"><span
      class="twowords">Mobilidade Urbana</span></a>
      <a href="#" class="circle c1"
      style="top:255px;left:487px;"><span>Reciclagem</span></a>
      <a href="#" class="circle c1"
      style="top:185px;left:762px;"><span>Educação</span></a>
      
      <a href="#" class="circle c2"
      style="top:230px;left:50px;"><span>Finanças</span></a>
      <a href="#" class="circle c2"
      style="top:209px;left:326px;"><span>Gestao</span></a>
      <a href="#" class="circle c2"
      style="top:346px;left:388px;"><span>Cultura</span></a>
      <a href="#" class="circle c2"
      style="top:145px;left:518px;"><span>Esporte</span></a>
      <a href="#" class="circle c2"
      style="left:446px;"><span>Política</span></a>
      
      <a href="#" class="circle c3"
      style="top:88px;left:187px;"><span>Ciencia</span></a>
      <a href="#" class="circle c3"
      style="top:112px;left:856px;"><span>Pecuária</span></a>
      <a href="#" class="circle c3"
      style="top:348px;left:877px;"><span>Pesca</span></a>
      <a href="#" class="circle c3"
      style="top:110px;left:776px;"><span>Engenharia</span></a>
      <a href="#" class="circle c3"
      style="top:335px;left:664px;"><span>Comunicação</span></a>
      <a href="#" class="circle c3"
      style="top:200px;left:687px;"><span>Política</span></a>
      
      <a href="#" class="circle c4"
      style="top:166px;left:172px;"><span>Gestão</span></a>
      <a href="#" class="circle c4"
      style="top:342px;left:744px;"><span>Indígena</span></a>
      <a href="#" class="circle c4"
      style="top:209px;left:617px;"><span>Assistencia</span></a>
      
      <a href="#" class="circle c4"
      style="top:317px;left:332px;"><span>Tecnologia</span></a>
      <a href="#" class="circle c4"
      style="top:277px;left:422px;"><span>Turismo</span></a>
    </div>
  </div>
  \end{verbatim}
}

\newpage
\section{Folha de Estilo - CSS}
\label{Att:CSS}

{\tiny
  \begin{verbatim}
  @import url(../profile-base/style.css);
  
  /* Main style page */
  body {
      background: #FFF;
  }
  
  #link-go-content {
      display: none;
  }
  
  #wrap-1 {
      width: 100%;
      margin: 0px;
  }
  
  #theme-header {
      font-size: 12px;
  }
  
  #wrap-2 {
      max-width: 960px;
      margin: auto;
      border: 0px;
      margin-top: 160px;
  }
  
  #main-content-wrapper-1,
  #main-content-wrapper-2,
  #main-content-wrapper-3,
  #main-content-wrapper-4,
  #main-content-wrapper-5,
  #main-content-wrapper-6,
  #main-content-wrapper-7,
  #main-content-wrapper-8 {
      background-image: none;
      background: none;
  }
  
  /* Overwriting Brasil gov style */
  
  body {
    font-size: 12px;
  }
  
  label {
    font-weight: normal;
  }
  
  #portal-siteactions {
     padding-bottom: 5px;
     margin-top: -10px;
  }
  
  /* Bar Psocial Style */
  
  #user {
     top: -30px;
     font-size: 12px;
  }
  
  #user form {
      display:none;
  }
  
  #barra-brasil {
      box-shadow: 0px 0px 10px #DFDFDF inset;
      z-index: 9999;
      position: relative;
      width: 100%;
  }
  
  #barra-psocial {
      position: relative;
      height: 40px;
      margin: auto;
      background: url(images/barra-psocial-bg.png) repeat-x;
  }
  
  #barra-psocial li {
    float: left;
  }
  
  #assets-menu {
      background: #E8E8E8;
      top: 35px;
      left: 80px;
      min-width: 132px;
  }
  #assets-menu a {
      border: 1px solid #E8E8E8;
  }
  
  #search-header {
      position: relative;
      width: 960px;
      margin: auto;
  }
  #search-header .search-field {
      width: 200px;
      position: absolute;
      top: 100px;
      right: 20px;
      z-index: 999;
  }
  #search-header input.button.with-text {
      width: 22px;
      text-indent: -1000px;
      float: right;
      height: 30px;
      max-height: 30px;
      border: none;
      background-color: transparent;
  }
  #search-header #q {
      width: 170px;
      height: 25px;
      border-radius: 5px;
  }
  
  #categories_menu {
      max-width: 960px;
      margin: auto;
  }
  
  #cat_menu {
      background: url(images/logo-PS-barra.png) no-repeat center
  left;
      height: 40px;
      padding-left: 60px;
  }
  
  #cat_menu li {
      list-style: none;
      font-size: 12px;
      font-weight: bold;
      padding: 15px 20px;
      text-transform: uppercase;
  }
  
  #cat_menu li#category1 {
      color: #ba4a00
  }
  #cat_menu li#category2 {
      color: #3b7390
  }
  #cat_menu li#category3 {
      color: #643c67
  }
  #cat_menu li#category4 {
      color: #826938
  }
  #cat_menu li#category5 {
      color: #1D571F
  }
  #cat_menu li#category6 {
      color: #017b16
  }
  #cat_menu li#category7 {
      color: #1a55dd
  }
  #cat_menu li#category8 {
      color: #753900
  }
  #cat_menu li#category9 {
      color: #56762b
  }
  #cat_menu li#category10 {
      color: #3867b7
  }
  #cat_menu li#category11 {
      color: #00439e
  }
  #cat_menu li#category12 {
      color: #00A0DB
  }
  #cat_menu li#category13 {
      color: #AD6500
  }
  #cat_menu li#category14 {
      color: #DE9200
  }
  
  /* Search Button */
  #search-button a {
      display: inline-block;
      width: 29px;
      height: 25px;
      margin-right: 3px;
      margin-top: 10px;
  }
  
  #search-button a:hover {
      opacity: 0.6;
  }
  
  #search-button #sb_search {
  background-image: url(images/search.png);
  background-size: 100% 100%;
  }
  
  #search-button span { display: none; }
  
  /* Social Buttons */
  #social-buttons a {
      width: 18px;
      height: 20px;
      margin-right: 3px;
      margin-top: 10px;
  }
  
  #social-icons a {
      width: 19px;
      height: 19px;
      display: inline-block;
      background-repeat: no-repeat;
  }
  
  #social-icons a:hover {
      opacity: 0.6;
  }
  
  #social-icons #sb_face {
  background-image: url(images/icone-facebook.png);
  }
  
  #social-icons #sb_tweet {
  background-image: url(images/icone-twitter.png);
  }
  
  #social-icons #sb_youtb {
  background-image: url(images/icone-youtube.png);
  }
  
  #social-icons #sb_flickr {
  background-image: url(images/icone-flickr.png);
  }
  
  #social-icons span { display: none; }
  
  /* Top links */
  #top-links {
    padding-bottom: 3px;
  }
  
  #top-links a {
      display: inline-block;
      font-size: 10px;
      margin-right: 3px;
      margin-top: 15px;
      color: #E7F1E9;
      text-decoration: none;
      text-transform:uppercase;
  }
  
  #top-links a:hover {
      opacity: 0.6;
  }
  
  #top-links #link-faq {
      border-right: 1px solid #E7F1E9;
      padding-right:5px;
  }
  
  /* Title Header */
  
  #content .main-block h1,
  #content .main-block h2,
  #content .main-block h3,
  #content .main-block h4 {
      font-family:  'Open Sans', Arial, Helvetica, sans-serif;
  }
  
  #content .main-block h1,
  #not-found h1,
  #access-denied h1 {
      color: #2C67CD;
      font-size: 1.7em;
      padding: 7px 0;
      margin-left: 0 !important;
      padding-left: 0.3em;
      border-bottom: 1px solid #CCCCCC;
      border-top: 2px solid #172838;
  }
  
  #content .title {
      font-weight: normal; !important;
      padding-right: 70px;
  }
  
  #content .main-block h1 {
      font-size: 2.3em !important;
      font-weight: bold !important;
  }
  
  #content .main-block h2 {
      font-size: 1.8em !important;
      min-height: 48px;
  }
  
  #content .main-block h3 {
      font-size: 1.5em !important;
      min-height: 48px;
  }
  
  #content .main-block h4 {
      font-size: 1.3em !important;
      min-height: 48px;
  }
  
  div#article-parent {
      position: absolute;
      right: 0px;
      top: 0px;
  }
  
  /* Menu List left */
  
  #content .box-2 .block-title {
      font-size: 12px;
      text-align: left;
      border-top: 4px solid #757575;
      background: #eeefff;
      border-bottom: none;
      color: #757575;
      padding: 8px 8px 24px 10px;
      text-transform: uppercase;
      margin: 0;
  }
  
  #content .box-2 .link-list-block li a.link-this-page {
      width: auto;
      border-right: none;
      font-weight: bold;
      color: #436976 !important;
      border-top: 2px solid #64946e !important;
      border-bottom: 2px solid #64946e !important;
      background-color: #eeefff;
      border-radius: 0px;
  }
  
  #content .box-2 .link-list-block li a {
      font-size: 14px;
      line-height: 1em;
      color: #436976;
      background-color: #FFF;
      border-radius: none;
      padding: 0.6em 1.5em;
  }
  
  #content .box-2 .link-list-block li a:hover {
      background-color: #FFF;
      color: #000;
  }
  
  #content .box-2 .link-list-block li {
      border-bottom: 1px solid #ddd;
      border-top: none;
      padding: 0;
      margin: 0;
  }
  
  /* Blocks main on home */
  
  .action-home-index .main-block {
      display: none;
  }
  
  .action-home-index .box-2 {
      display: none;
  }
  
  .action-home-index .box-1 {
      margin-left: 0px;
  }
  
  /* Blocks main style */
  
  .action-home-index #content .box-1 .block-title {
      padding: 0.6em 1.5em;
      color: #444;
      text-transform: uppercase;
      font-variant: normal;
  }
  
  .action-home-index #content .box-1 .main-block {
      background: #FFF;
      width: 100%;
      clear: both;
  }
  
  .action-home-index #content .box-1 .article-block {
      background: none;
  }
  
  #content .blog-post .title {
      text-align: left;
  }
  
  /* Box styles */
  .box-1 {
      Xmargin: 0 0 0 170px;
  }
  
  .action-home-index .box-2 {
      width: 160px;
  }
  
  .action-home-index #content .box-3 .block {
      width: 210px;
      margin: 0px 0px 0px 30px;
      float: left;
  }
  
  /* Editorial Area */
  
  #content .box-1 .news-area {
      width: 32%;
      margin-right: 10px;
  }
  
  #content .box-1 a {
      text-decoration: none;
  }
  
  #content .box-1 .news-area h3 {
      background: #DFDFDF;
      text-decoration: none;
      line-height: 40px;
      height: 40px;
      min-height: 40px;
      border-top: 8px solid #545454;
      padding-left: 10px;
      text-transform: uppercase;
      font-weight: normal;
      font-size: 20px;
  }
  
  #content .box-1 .news-area a.news-see-more {
      position: relative;
      border-top: 3px solid #545454;
      background-color: #DFDFDF;
      width: auto;
      display: block;
      text-align: right;
      padding: 5px 15px;
  }
  
  #content .box-1 .news-area ul {
      background: #FFF;
      border: none;
      background-image: none;
  }
  
  .blog-post {
      background: none !important;
      margin: 0px;
  }
  
  /* Block Display Content style */
  
  .box-1 .display-content-block {
      font-size: 14px;
  }
  
  .box-1 .display-content-block h2 {
      font-size: 24px;
      font-weight: bold;
  }
  
  /* Block Article style */
  
  .action-home-index #content .box-1 .article-block {
      width: 100%;
      font-size: 14px;
  }
  
  .action-home-index #content .box-1 .article-block .block-title {
      display: none;
  }
  
  .action-home-index #content .box-1 .article-block h2 {
      font-size: 34px;
      font-weight: bold;
  }
  
  .action-home-index #content .box-1 .article-block .read-more {
      float: right;
  }
  
  .action-home-index #content .box-1 .article-block .read-more a {
      display: block;
      height: 39px;
      width: 161px;
      text-align: center;
      padding: 7px;
      background: url(images/button-read-more2.png) no-repeat;
  }
  
  .action-home-index #content .box-1 .article-block .read-more a,
  .action-home-index #content .box-1 .article-block .read-more a:hover {
     color: transparent !important;
  }
  
  .action-home-index #content .box-1 .article-block .read-more
  a:first-child {
      display: none;
  }
  
  /* Block My Network style */
  
  .action-home-index #content .my-network-block {
      background: #eeefff;
  }
  
  .action-home-index #content .my-network-block .block-title {
      background: #757575;
      padding: 0.6em 1.5em;
      color: #FFF;
  }
  
  .action-home-index #content .my-network-block ul {
      padding: 10px 0px 10px 20px;
  }
  
  .action-home-index #content .my-network-block .my-network-actions
  a.button {
      width: 100px;
      display: block;
      margin-bottom: 5px;
  }
  
  /* Blocks profiles and enterprises */
  .recent-documents-block ul {
      padding: 0px 0px 0px 30px;
  }
  
  .box-1 .menu-submenu {
      bottom: 80px;
      right: -40px;
  }
  
  .box-1 .common-profile-list-block .vcard {
      border-radius: 0px;
      -moz-border-radius: 0px;
  }
  
  .box-1 .common-profile-list-block .profile_link {
      width: 65px;
      height: 80px;
  }
  
  .box-1 .common-profile-list-block span {
   width: auto;
  }
  
  .box-1 .common-profile-list-block .profile-image {
      display: block;
      height: auto;
      width: auto;
      text-align: center;
      margin: 0px;
      padding: 0px;
      padding-bottom: 5px;
  }
  
  .box-1 .common-profile-list-block {
      margin-left: 15px;
  }
  
  .action-home-index #content .communities-block .block-title,
  .action-home-index #content .display-people-block .block-title {
      display: none;
  }
  
  #content .display-people-block .banner-span {
      background-color: #f15921;
      color: #FFF;
      font-size: 24px;
      font-weight: normal;
  }
  
  .box-1 .display-people-block .banner-span,
  .common-profile-list-block .vcard a {
      width: 108px;
      height: 57px;
  }
  
  #content .display-people-block .vcard a.profile_link {
      height: 52px;
      margin: 0;
      width: 52px;
  }
  
  .box-1 .block-footer-content {
      clear: both;
      text-align: right;
      font-size: 11px;
      color: #000;
      position: relative;
      padding: 5px;
  }
  
  #content .box-1 .block-footer-content a {
      position: relative;
  }
  
  /* Blocks tags */
  .action-home-index #theme-footer .block.tags-block {
      background: #034811;
      min-height: 50px;
      padding: 20px 0;
      width: 100%;
      margin-bottom: 0px;
      background-image: url(images/bg-fundo-verde-tags.png);
  }
  
  .action-home-index #theme-footer .block.tags-block .block-inner-1 {
      max-width: 960px;
      margin: auto;
  }
  .action-home-index #theme-footer .block.tags-block .block-title {
      display: none;
  }
  
  .action-home-index #theme-footer .tags-block .block-footer-content a {
      color: #FFFFFF;
  }
  
  .action-home-index #theme-footer .block.tags-block a,
  .action-home-index #theme-footer .block.tags-block a:visited,
  .action-home-index #theme-footer .block.tags-block a:hover {
      color: #E7F1E9;
  }
  
  .action-home-index #theme-footer .block.tags-block a:hover {
      text-decoration: underline;
  }
  
  /* Menu List footer */
  
  #content .box-3 .block-title {
      font-size: 12px;
      text-align: left;
      border-top: none;
      background: #FFF;
      border-bottom: none;
      color: #757575;
      padding: 5px;
      text-transform: uppercase;
      margin: 0;
  }
  
  #content .box-3 .link-list-block li a.link-this-page {
      width: auto;
      border-right: none;
      font-weight: bold;
      background-color: #eeefff;
      border-radius: 0px;
  }
  
  #content .box-3 .link-list-block li a {
      font-size: 14px;
      line-height: 1em;
      color: #545454;
      background-color: #FFF;
      border-radius: none;
      padding: 0.6em 1.5em;
  }
  
  #content .box-3 .link-list-block li a:hover {
      background-color: #FFF;
      color: #436976;
  }
  
  #content .box-3 .link-list-block li {
      border-bottom: none;
      border-top: none;
      padding: 0;
      margin: 0;
  }
  
  .agenda-item a {
    color: black;
  }
  .agenda-item a:visited {
    color: black;
    font-weight: normal;
  }
  .agenda-item a:hover {
    color: black;
    text-decoration: underline !important;
  }
  
  #box-organizer .block-target { clear: both; }
  
  /* Custom Icons */
  
  .box-2 .link-list-block ul li .icon-ok,
  .box-2 .link-list-block ul li .icon-eyes,
  .box-2 .link-list-block ul li .icon-edit,
  .box-2 .link-list-block ul li .icon-photos {
      background-position: 0px -325px !important;
      height: 27px;
      padding: 13px 0px 0px 45px !important;
      margin-top: 5px;
      margin-bottom: 5px;
  }
  
  .box-2 .link-list-block ul li .icon-ok {
      background-position: 0px -165px !important;
  }
  .box-2 .link-list-block ul li .icon-eyes {
      background-position: 0px -125px !important;
  }
  .box-2 .link-list-block ul li .icon-edit {
      background-position: 0px -245px !important;
  }
  
  /* Footer */
  
  #theme-footer {
      width: 100%;
  }
  
  #footer-logos {
      background: #00420C;
      max-width: 100%;
      padding: 2em 0;
      height: 49px;
  }
  
  #footer-logos div,
  #footer-license {
      max-width: 960px;
      margin: 0 auto;
  }
  
  #footer-logos a {
    display: block;
    height: 49px;
    float: left;
  }
  
  #footer-logos span {
    display: none;
  }
  
  #footer-logos .logo-acesso {
    background: transparent url(images/acesso-a-informacao.png) center
  center no-repeat;
    width: 107px;
  }
  
  #footer-logos .logo-brasil {
    background: transparent url(images/brasil.png) center center
  no-repeat;
    width: 153px;
  }
  
  #footer-logos .logo-sgpr {
    background: transparent url(images/sgpr.png) center center no-repeat;
    width: 187px;
    margin-right: 30px;
  }
  
  #footer-logos .institucionais {
    float: right;
  }
  
  #footer-license {
    text-align: left;
  }
  
  /* padrao da barra verde - estatistica */
  .action-home-index #content .box-1 .statistics-block {
     width: 100%;
     display: inline-block;
     margin: 0px;
     padding: 28px 0px 15px 0px;
  }
  
  .action-home-index .statistics-block .block-title {
     display: none;
  }
  
  .action-home-index .statistics-block-data ul{
          list-style:none;
  }
  
  .action-home-index .statistics-block-data ul li{
          float:left;
          width:146px;
          margin:0 20px 20px 0px;
          text-align:center;
          color:#fff;
  }
  
  .action-home-index .statistics-block-data ul li .amount{
          font-size:40px !important;
          font-weight:bold;
          width:136px;
  }
  
  .action-home-index .statistics-block-data ul li .label{
          font-size:14px !important;
          clear:both;
          float:left;
          width:136px;
          color:#98baa5;
  }
  
  #header input.searchButton {
    background-image: url(images/search-button.gif);
  }
  
  /* HOME */
  
  .action-home-index #wrap-2 {
    max-width: none;
    padding: 0px;
  }
  
  .action-home-index #boxes {
    margin-top: 0px;
  }
  
  .action-home-index #content {
    margin: 0px;
  }
  
  .action-home-index .home-blocks-inner {
    max-width: 960px;
    margin: auto;
  }
  
  .action-home-index #content-blocks {
    margin-top: 30px;
    clear: both;
  }
  
  .action-home-index #content-blocks .home-blocks-inner {
    text-align: center;
  }
  
  .action-home-index #content .box-1 .display-content-block {
      width: 300px;
      display: inline-block;
      text-align: left;
  }
  
  .action-home-index #content #content-blocks-inner {
      max-width: 960px;
  }
  
  #content .box-1 #content .statistics-block li,
  #content .box-1 #content .display-content-block li {
      list-style: none;
      margin: 0px;
  }
  
  .action-home-index #content .communities-block,
  .action-home-index #content .display-people-block {
      display: inline-block;
      width: 50%;
      vertical-align: top;
      margin-top: 30px;
  }
  
  .action-home-index #content .box-1 .block {
      margin-right: 0px;
  }
  
  .action-home-index #statistics-blocks {
          background:#0c763e;
  }
  
  .action-home-index .statistics-block {
          padding:10px 0 25px 0;
  }
  
  .action-home-index #circles-blocks {
      width: 100%;
      background: url(images/bg-palacio-do-planalto.jpg);
      padding-top: 20px;
  }
  
  .action-home-index #block.track-list-blocks {
      width: 100%;
      background: url(images/bg-pessoas.jpg);
  }
  
  .action-home-index #profile-blocks {
      background: url(images/bg-linhas-cinza.png);
  }
  
  \end{verbatim}
}


\end{document}
